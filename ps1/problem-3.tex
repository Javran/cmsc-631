\documentclass[11pt,a4paper]{article}
\usepackage[margin=1.5in]{geometry}
\usepackage{amsmath}
\usepackage{amsthm}
\usepackage{amssymb}
\usepackage{indentfirst}
\usepackage{bussproofs}
\usepackage{minted}
\usepackage{graphicx}
\usepackage{titling}

\title{Haskell is my favorite programming language}
\author{Javran Cheng}
\date{}
\setlength{\droptitle}{-0.8in}
\begin{document}

\maketitle
\thispagestyle{empty}

\paragraph{Haskell has an expressive syntax}
Haskell comes with expressive syntaxs to achieve
pattern matching and list comprehension, [\ref{lst:fib-and-lh}]
which saves many lines of code but preserves the readability.

\begin{listing}
\begin{minted}[mathescape]{haskell}
-- pattern matching: Fibonacci numbers
fib 0 = 0
fib 1 = 1
fib n = fib (n-1) + fib (n-2)

-- list comprehension: all Pythagorean triples $\in \mathbb{Z}_{>0}^3$
[(a,b,c) | a <- [1..10] , b <- [1..10] , c <- [1..10]
         , a <= b , b <= c, a * a + b * b == c * c]
-- result: [(3,4,5),(6,8,10)]
\end{minted}
\caption{Example of pattern matching and list comprehension}
\label{lst:fib-and-lh}
\end{listing}

\paragraph{Haskell has a strong and static type system}
Haskell's type system is strong. In general the type of an operator
and its operands are required to be exactly matched and 
there is no way of forcing type conversion in Haskell,
thus the compiler will complain if you are doing something improper
(e.g. trying to use integers as strings). 
Moreover, Haskell is equipped with static typing, and therefore 
is capable of type inference.
Suppose you are writing a function \texttt{f} to sum up a list of integers,
you can simply write \texttt{f = foldl' (+) 0}, which is enough for Haskell
to work out \texttt{f}'s signature:\texttt{[Integer] -> Integer}.
One can in addition gain some intuition about
what a function is doing
by looking at its name and type signature. Take
\texttt{filter :: (a -> Bool) -> [a] -> [a] }
as an example, we can tell from the type signature that
a predicate and a list is requied, and we will get a list of the 
same type in return. By looking at its name, we can further tell 
that the function should work like a ``filter''
using the predicate given as a guide.

\paragraph{Haskell enables explicit control on side effects}
In many other languages, side effects can be created everywhere
by doing I/O or mutate global variables, which make some procedures
error-prone and hard to reason.
By introducing the concept of monads, Haskell models side effects using
its type system so that a value are tied to the side effect it causes.
You can for example get user input using \texttt{getLine}, and in return
you get a value of \texttt{IO String}, and there is no way for this string
to live without \texttt{IO}.  
Therefore we have controls over side effects and
functions are guaranteed to produce the same result
as long as they are called with exactly the same arguments,
which is good for unit testing and debugging.

\end{document}
